\newcommand{\TemplateEnergieerhaltung}[1]{
\begin{longtable}{|>{\RaggedRight}p{\linewidth}|} \hline
{\bfseries Gesetz von der Erhaltung der Energie}\\ \hline
{\bfseries Albert Einstein (14.3. 1879 - 18.4.1955)}: Umwandlung von Energie in Masse und von Masse in Energie ist möglich.\\ 
$E = m \cdot c^2$ (c = Lichtgeschwindigkeit = 300.000 km/s)\\ \hline
{\bfseries 
Bei einer chemischen Reaktion ist die Summe aus Masse und Energie der Ausgangsstoffe gleich der Summe aus Masse und Energie der Endstoffe.
}\\\hline
Wird Energie frei, tritt ein unwägbar kleiner Massenverlust auf. Wird Energie investiert, tritt Massenzunahme auf. Dieses kann allerdings mit herkömmlichen Waagen nicht gemessen werden. \\ \hline
\end{longtable}
}

\newcommand{\TemplatePeriodensystem}[1]{
Hier sollte das Periodensystem stehen. Ein solches wird sehr wahrscheinlich von Orlando Camargo Rodriguez frei zur Verfügung gestellt werden. Dateiname: tabela_periodica.tex ist bereits online. Lizenz aber noch nicht genau genug definiert.
}

\newcommand{\TemplateMassenerhaltung}[1]{
\begin{longtable}{|>{\RaggedRight}p{\linewidth}|} \hline
{\bfseries Gesetz von der Erhaltung der Masse}\\ \hline
{\bfseries Antoine Lavoisier (1743 - 1794)}: Rien ne se perd, rien ne se crée\\ 
Die Gesamtmasse ändert sich bei chemischen Reaktionen (im Rahmen der Messgenauigkeiten) nicht.\\ \hline
Masse der Ausgangsstoffe=Masse der Produkte \\ \hline
\end{longtable}
}

\newcommand{\TemplateDaltonsAtomhyposthese}[1]{
\begin{longtable}{|>{\RaggedRight}p{\linewidth}|} \hline
\begin{enumerate}
\item Materie besteht aus extrem kleinen, bei Reaktion ungeteilt bleibenden Teilchen, den Atomen.
\item Die Masse der Atome eines bestimmten Elements sind gleich (alle Atome eines Elements sind gleich). Die Atome verschiedener Elemente unterscheiden sich in ihren Eigenschaften (zum Beispiel in Größe, Masse, usw.).
\item Es existieren so viele Atomsorten wie Elemente.
\item Bei chemischen Reaktionen werden Atome in neuer Kombination vereinigt oder voneinander getrennt.
\item Eine bestimmte Verbindung wird von den Atomen der betreffenden Elemente in einem bestimmten, einfachen Zahlenverhältnis gebildet.
\end{enumerate}
\\ \hline
\end{longtable}
}

\newcommand{\TemplateUnveraenderlicheMassenverhaeltnisse}[1]{
\begin{longtable}{|>{\RaggedRight}p{\linewidth}|} \hline
{\bfseries Gesetz der unveränderlichen Massenverhältnisse}\\ \hline
Louis Proust (1799) \\ \hline
Bei chemischen Reaktionen, also Vereinigung beziehungsweise Zersetzung, reagieren die Reinstoffe immer in einem von der Natur vorgegebenen festen Verhältnis miteinander.
\\ \hline
\end{longtable}
}
